%%
%% 研究報告用スイッチ
%% [techrep]
%%
%% 欧文表記無しのスイッチ(etitle,eabstractは任意)
%% [noauthor]
%%

%\documentclass[submit,techrep]{ipsj}
\documentclass[submit,techrep,noauthor]{ipsj}
\usepackage[dvipdfmx]{graphicx}
\usepackage{latexsym}
\usepackage{cite}
\usepackage[hyphens]{url}
\usepackage[dvipdfmx,hidelinks]{hyperref}
\def\Underline{\setbox0\hbox\bgroup\let\\\endUnderline}
\def\endUnderline{\vphantom{y}\egroup\smash{\underline{\box0}}\\}
\def\|{\verb|}
%

%\setcounter{巻数}{59}%vol59=2018
%\setcounter{号数}{10}
%\setcounter{page}{1}


\begin{document}
\title{情報処理学会 電子化知的財産・社会基盤研究会\\研究報告用テンプレート}

\etitle{Template for Tech Report for Electronic Intellectual Property Research Group,
Information Processing Association of Japan}

\affiliate{jilis}{一般財団法人情報法制研究所, Japan Institute of Law and Information Systems}

\author{猪谷 誠一}{Seiichi IGAYA}{jilis}[oasis\_adhim@proton.me]

\begin{abstract}
このテンプレートは情報処理学会が公開している研究報告用LaTeXテンプレートを
Zoteroでの文献管理を前提に、電子化知的財産・社会基盤(EIP)研究会用に拡張したものです。
和文誌、日米欧の法律、日欧の判決等の引用をサポートする拡張を行いました。
\end{abstract}

\begin{eabstract}

This template is an extension of the techreport template published 
by the Information Processing Society of Japan for the use of
Electronic Intellectual Property Research Group.
The modification covers the citation of japanese literature, laws in Japan, US, and the EU, cases in Japan and the CJEU.
\end{eabstract}


\begin{jkeyword}
\LaTeX, テンプレート, 情報処理学会
\end{jkeyword}


\begin{ekeyword}
\LaTeX, template, IPSJ
\end{ekeyword} 

\maketitle

\section{はじめに}
このテンプレートは情報処理学会電子化知的財産・社会基盤(EIP)研究会の研究報告で使うために、
同学会が公開している\LaTeX テンプレートを拡張したものです。
元のテンプレートはEIP研究会が参照する文献の扱いに以下のような課題がありました。
\begin{enumerate}
    \item 文献管理ソフトと併用すると和文文献の著者名が「名 姓」になってしまう。
    \item bstファイルに判例や法律といった文献タイプの関数が定義されていない。
\end{enumerate}
作者は自分用に上記に対応するbstファイルを作っていましたが、
\LaTeX テンプレートがないことは文理の垣根を越えた研究を掲げるEIP研究会に
理数系の研究者が参加するハードルになっているように思えたため、
このテンプレートを提供することにしました。

\section{想定環境}
この文書では、読者に基本的な\LaTeX の使い方の知識があることを想定して、
法学関連の文献情報を適切に引用する方法を説明しています。
\LaTeX の使用に不安のある方は\cite{okumuraLaTeX2023}などをご覧になるとよいでしょう。

\subsection{コンパイル環境}
このテンプレートはOverleaf (\url{https://overleaf.com/})での使用を前提としています。
元テンプレートがplatexを想定しているためか、著者のローカル環境(Macintosh + lualatex)ではコンパイルが通らないためです。

\subsection{\LaTeX パッケージ}
URLの折り返しのため、urlパッケージとhyperrefパッケージが必須です。
urlパッケージにはhyphensオプションを指定、
hyperrefパッケージには、hidelinksオプションと、dviからpdfを生成するソフトウェア名をオプションと指定してください。
なお、urlパッケージをhyperrefパッケージより前に指定する必要があります。
この\TeX ファイルのプリアンブルを参考にしてください。

\subsection{文献管理ソフト}
以下ではZotero (\url{https://www.zotero.org/})の使用を想定して説明します。
\subsubsection{Zoteroの準備}
準備として、マイ・ライブラリ下に論文や研究プロジェクトごとにコレクション
(音楽再生ソフトでの「プレイリスト」のようなもの)を作っておきます。
左ペインの左上隅にある、フォルダアイコンに$\oplus $が付いているアイコン
(図\ref{fig:doi}左の「マイ・ライブラリ」の左上にあるもの)を
クリックするとコレクションを新規作成できます。

\subsubsection{Better BibTexプラグインの設定}
また、判決や法令など、国ごとに異なる引用形式を扱うため、
Zoteroの「権利」フィールドに国などの情報を持たせることにしています。
しかし、インストール直後のZoteroは権利フィールドをエクスポートすることができないので、
Better BibTexプラグインをインストールした上で
その設定を行う必要があります。手順は以下の通りです。

\begin{enumerate}
\item Better BibTeXのページ(\url{https://retorque.re/zotero-better-bibtex/})に
アクセスして左ペインの下の方にある「Download」をクリック、
遷移先githubページにある「zotero-better-bibtex-バージョン番号.xpi」をダウンロードします。
\item ダウンロードできたらZoteroの「ツール→Plugins→右上の歯車→Install Plugin From File...」でダウンロードしたxpiファイルを選択してBetter BibTeXプラグインをZoteroにインストールします。
\item Zoteroの「環境設定→Better BibTeX→postscript」に以下のテキストを入力。
    \begin{verbatim}
    if (Translator.BetterBibLaTeX) {
      tex.add({ name: 'rights', 
        value: zotero.rights});
    }
    \end{verbatim}
\end{enumerate}
これでコレクションをエクスポートしたときにrightsフィールドも含まれるようになったはずです。

\section{文献管理ソフトへの入力}

\textbf{重要:}
著者名の「名 姓」と「姓 名」の振り分けはZoteroの「言語」フィールド
(bibitemのlangidがjapaneseか否か)が出力する
langidの値で判定しています。
なので、著者名を「姓 名」にしたい場合には、論文中で使われている言語によらず
言語フィールドにはjaを入力してください。

著者名を足したいときは、著者名入力欄の右にマウスオーバーすると
$\ominus $と $\oplus $ が表示されるので、
$\oplus $をクリックすれば追加の著者欄が出てきます。
著者名の入力フィールドが2つに分かれている場合はauthorフィールドに「姓」「名」の順で、
1つの場合にはinstitutionフィールドに値が入ります。
著者名が「名 姓」の人であっても入力欄は左が姓で右が名なので注意しましょう。
この2つを切り替えるには著者名入力欄の右にある長方形をクリックします。
「著者名」の文字をクリックすると「編者」「訳者」等も選べます。
ただ、「編著者」というのはないため「著者」に統一することになると思います。

また、Zoteroのcitation keyの自動生成は結構長いものを生成するので、
メインページで文献を右クリック→Better BibTeX→Change BibTeX Key...で手動設定すること
をおすすめします。

\subsection{和文論文}

DOI等の文献番号が付されている場合には、
メインパネル上部の杖のアイコン(図\ref{fig:doi})をクリックして入力すれば
自動でインポートされます。
ですが、ほとんどの場合言語フィールドは空なので適宜入力してください。

\begin{figure}
    \centering
    \includegraphics[width=0.8\linewidth]{./fig/doi.png}
    \caption{DOI等でインポートするためのアイコン。}
    \label{fig:doi}
\end{figure}


図\ref{fig:article_j}のように入力すると
(DOIより下は自動で生成されるので入力の必要はありません。以下同様。)、\cite{itakuraDataSetLicense2022}のように
出力されます。
必須のフィールドは
\begin{itemize}
    \item タイトル
    \item 著者名
    \item 雑誌名
    \item 出版年月日(年だけでOK)
    \item 言語
\end{itemize}
です。これらに加えて巻とページ数も普通は入ると思います。

\begin{figure}
    \centering
    \includegraphics[width=0.8\linewidth]{./fig/article_j.png}
    \caption{和文論文での入力例。}
    \label{fig:article_j}
\end{figure}

\subsection{和文書籍}
Zoteroの文献番号インポート機能(図\ref{fig:doi})はISBNにも対応していますが、
ISBNから引っ張ってこれる和文書籍の情報はフォーマットが統一されていないようなので
手動で入力した方が良さそうです(図\ref{fig:book_j})。

必須のフィールドは
\begin{itemize}
    \item タイトル
    \item 著者名
    \item 出版社名
    \item 出版年月日(年だけでOK)
    \item 言語
\end{itemize}

です。これによって\cite{retrospectHoribe2023}のように出力されます。

\begin{figure}
    \centering
    \includegraphics[width=0.8\linewidth]{./fig/book_j.png}
    \caption{和文書籍での入力例。}
    \label{fig:book_j}
\end{figure}

\subsection{和文書籍の章}

書籍とほとんど変わりませんが、必須のフィールドは
\begin{itemize}
    \item 章のタイトル(題名)
    \item 著者名
    \item 書籍名
    \item 出版社名
    \item 出版年月日
    \item 言語
\end{itemize}
が必須です。また、編者名も入れるのが良いでしょう(図\ref{fig:book-chapter_j})。
これによって\cite{Horibe-InternationalCoherence-2010}のように出力されます。

\begin{figure}
    \centering
    \includegraphics[width=0.8\linewidth]{./fig/book-chapter_j.png}
    \caption{和文書籍の章での入力例。}
    \label{fig:book-chapter_j}
\end{figure}

\subsection{和文学会発表}

アイテムを「学会発表」にした上で、言語を「ja」にします(図\ref{fig:inproceedings_j})。
必須フィールドは
\begin{itemize}
    \item 題名
    \item 著者名
    \item 出版年月日
    \item 紀要名
    \item 巻
    \item 言語
\end{itemize}
です。学会名を入力することもできますが、
海外のプロシーディングスを引用する際には学会名より紀要名を利用する(紀要名に学会名が入っている)ことが多いようなので、
現在「学会名」は引用に使うようにはなっていません。
これにより\cite{igayaIPAddress2025}のように出力されます。

\begin{figure}
    \centering
    \includegraphics[width=0.8\linewidth]{./fig/proceedings_j.png}
    \caption{学会発表の入力例。}
    \label{fig:inproceedings_j}
\end{figure}

\subsection{法}
アイテムは「法律」を選択します。
権利欄への入力内容で引用形式を振り分けています。現在対応している法と権利への入力内容は、
\begin{description}
    \item [日本法] Jlaw
    \item [米国連邦法] USlaw
    \item [米国州法] USstatelaw
    \item [EU法] EUlaw
\end{description}
です。

\subsubsection{日本法}
以下を入力すると(図\ref{fig:jlaw})、
\begin{itemize}
    \item 法令名
    \item 法律番号
    \item 言語はjaと入力
    \item 権利にJlawと入力
\end{itemize}
\cite{appi2003}のように引用されます。

\begin{figure}
    \centering
    \includegraphics[width=0.8\linewidth]{./fig/law_J.png}
    \caption{日本法(個人情報保護法)の入力例。}
    \label{fig:jlaw}
\end{figure}

\subsubsection{米国連邦法}

以下を入力すると(図\ref{fig:uslaw})、
\begin{itemize}
    \item 法令名
    \item コード (United States CodeならU.S.C.など)
    \item 法律番号
    \item 施行期日 (法典の版)
    \item 権利にUSlawと入力
\end{itemize}

\cite{CopyrightAct19762012}のように引用されます。

\begin{figure}
    \centering
    \includegraphics[width=0.8\linewidth]{./fig/law_US.png}
    \caption{米国連邦法(著作権法)の入力例。}
    \label{fig:uslaw}
\end{figure}

\subsubsection{米国州法}
以下を入力すると(図\ref{fig:usstatelaw})、
\begin{itemize}
    \item 法令名
    \item コード
    \item 施行期日
    \item ページ数 (冒頭のセクション番号)
    \item 権利にUSstatelawと入力
\end{itemize}
\cite{ccpa2018}のように引用されます。

\begin{figure}
    \centering
    \includegraphics[width=0.8\linewidth]{./fig/law_USstate.png}
    \caption{米国州法(CCPA)の入力例。}
    \label{fig:usstatelaw}
\end{figure}

\subsubsection{EU法}

以下を入力すると(図\ref{fig:eulaw})、
\begin{itemize}
    \item 法令名
    \item 法律番号 (EU官報の巻号)
    \item 施行期日 (EU官報の日付)
    \item ページ数 (EU官報の最初の頁)
    \item 権利にEUlawを入力
\end{itemize}
\cite{gdpr2016}のように引用されます。
EU官報の情報がない場合には、法律番号を空欄にすればスキップされます。

\begin{figure}
    \centering
    \includegraphics[width=0.8\linewidth]{./fig/law_EU.png}
    \caption{EU法(GDPR)の入力例。}
    \label{fig:eulaw}
\end{figure}

\subsubsection{その他の国・地域の法}

権利に上記以外の値を入力(大文字・小文字違いや空欄を含む)すると(図\ref{fig:prclaw})、
デフォルトの引用形式
\cite{pipl2021}
が出力されます。

\begin{figure}
    \centering
    \includegraphics[width=0.8\linewidth]{./fig/law_PRC.png}
    \caption{その他国・地域の法(中華人民共和国個人情報保護法)の入力例。権利に入力された「PRClaw」は無視され、
    デフォルトの引用形式で出力される。}
    \label{fig:prclaw}
\end{figure}


\subsection{判決など}

アイテムを「訴訟」にします。
権利やレポーター、リポーター巻への入力内容で引用形式を振り分けています。

\subsubsection{日本の判決や審決}
公刊物に登載されている判決や審決の場合、図\ref{fig:case_japan_reported}のように入力すると、
\cite{niftyserveCase2002}のように出力されます。
必須のフィールドは
\begin{itemize}
    \item 事件名(裁判所名称 裁判形式 判決年月日までを記入)
    \item 決定期日(西暦で入力。文献ソフトでの管理用です)
    \item レポーター (掲載誌名)
    \item リポーター巻(掲載誌の号)
    \item 最初のページ(掲載誌の最初のページ)
    \item 歴史に「case」
    \item 権利に「Jcase」
\end{itemize}
また、読者の便宜のために事件の通称を付したい場合は、「題名(略)」に入力します。
事件整理番号に事件番号を入れても構いません。

\begin{figure}
    \centering
    \includegraphics[width=0.8\linewidth]{./fig/case-japan-reported.png}
    \caption{日本の判決や審決で公刊物に登載されているものの入力例。}
    \label{fig:case_japan_reported}
\end{figure}

公刊物に登載されていない判決の場合、図\ref{fig:case_japan_notreported}のように入力すると、
\cite{vtuber2021}のように、「公刊物未登載」の文言付きで出力されます。
必須のフィールドは
\begin{itemize}
    \item 事件名(裁判所名称 裁判形式 判決年月日までを記入)
    \item 決定期日(西暦で入力。文献ソフトでの管理用です)
    \item 事件整理番号
    \item 歴史に「case」
    \item 権利に「Jcase」
\end{itemize}
また、読者の便宜のために事件の通称を付したい場合は、「題名(略)」に入力します。
なお、公刊物には掲載されていないが裁判所等のウェブページで閲覧できる場合にはURLを入力しておくと
「公刊物未登載」の文言が「裁判所等のウェブページ参照」に差し替わります。

\begin{figure}
    \centering
    \includegraphics[width=0.8\linewidth]{./fig/case-japan-unreported.png}
    \caption{日本の判決や審決で公刊物に登載されていないものの入力例。}
    \label{fig:case_japan_notreported}
\end{figure}

判例DBに掲載されている場合は、
\begin{itemize}
    \item 事件名(裁判所名称 裁判形式 判決年月日までを記入)
    \item 決定期日(西暦で入力。文献ソフトでの管理用です)
    \item レポーター(DB名称)
    \item 最初のページ(文献版号)
    \item 歴史に「case」
    \item 権利に「Jcase」
\end{itemize}
とすることもできます(図\ref{fig:case_japan_db})。
これにより\cite{vtuber2020}のように引用されます。
\begin{figure}
    \centering
    \includegraphics[width=0.8\linewidth]{./fig/case-japan-db.png}
    \caption{日本の判決でDBに掲載されているものの入力例。}
    \label{fig:case_japan_db}
\end{figure}

\subsubsection{CJEUの判決}

図\ref{fig:case_cjeu}のように入力すると\cite{PatrickBreyerBundesrepublik2016}のように出力されます。
必須のフィールドは
\begin{itemize}
    \item 事件名
    \item 裁判所
    \item 決定期日
    \item 事件整理番号
    \item 最初のページに欧州判例法識別子(ECLI)
    \item 歴史に「case」
    \item 権利に「CJEUcase」
\end{itemize}
です。

\begin{figure}
    \centering
    \includegraphics[width=0.8\linewidth]{./fig/case-CJEU.png}
    \caption{欧州司法裁判所判決の入力例。}
    \label{fig:case_cjeu}
\end{figure}

\subsubsection{その他の国・地域の判決}
権利に「Jcase」「EUcase」以外の文字列(大文字・小文字違いや空欄も含む)を入力すると
(図\ref{fig:case_other})、jurisdictionのデフォルトの引用形式として
\cite{conseildetatBodyTemp2020}のように出力されます。
必須のフィールドは
\begin{itemize}
    \item 事件名
    \item 裁判所
    \item 歴史に「case」
    \item 決定期日
\end{itemize}
です。「事件整理番号」「最初のページ」にも対応しています。

\begin{figure}
    \centering
    \includegraphics[width=0.8\linewidth]{./fig/case-other.png}
    \caption{その他の国・地域判決の入力例。権利に入力された「Francecase」は無視され、
    デフォルトの引用形式で出力される。}
    \label{fig:case_other}
\end{figure}


\subsection{国会議事録}
\label{ssec:jdiet}
アイテムは「公聴会」を選択した上で、「歴史」に``parliament''と入力します。
「権利」フィールドの値で振り分けをしていますが、現状は日本(Japan)か否かしか見ておらず、
しかもいずれの場合でも出力形式は同じです。
ですが、今後対応する国・地域を増やす可能性があるため、現時点から「権利」には国・地域名を入力しておくことをおすすめします。

日本の国会議事録の場合は、「権利」に``Japan''という文字列を入力してください。
内部的には「訴訟」と同じくjurisdictionになってしまいます。
必須のフィールドは
\begin{itemize}
    \item 題名
    \item 立法機関
    \item 出版年月日(委員会開催日)
    \item 言語
    \item 歴史に「parliament」
    \item 権利に「Japan」
\end{itemize}
です(図\ref{fig:diet_j})。これによって\cite{sangi156appi8}のように出力されます。

\begin{figure}
    \centering
    \includegraphics[width=0.8\linewidth]{./fig/diet-japan.png}
    \caption{国会議事録の入力例。}
    \label{fig:diet_j}
\end{figure}

\subsection{ソフトウェア}

\begin{figure}
    \centering
    \includegraphics[width=0.8\linewidth]{./fig/software.png}
    \caption{ソフトウェアの入力例。}
    \label{fig:software}
\end{figure}

アイテムは「ソフトウェア」を選択します。
必須フィールドは
\begin{itemize}
    \item プログラマー
    \item 題名
\end{itemize}
で、これ以外に
\begin{itemize}
    \item バージョン
    \item 出版年月日
    \item URL
\end{itemize}
を入力することができ(図\ref{fig:software})、
\cite{ditigalScholarZotero2024}のように出力されます。


\subsection{ウェブページ}
アイテム名を「ウェブページ」にします。
なぜか図\ref{fig:doi}の丸をつけたアイコンの左にある
新規アイテム作成をクリックして現れるリスト中に「ウェブページ」がありませんが、
作成後に文献を選択して右ペインに出すと「アイテムの種類」から
「ウェブページ」を選べるようになります。
必須フィールドは

\begin{itemize}
    \item 題名
    \item URL
    \item アクセス日時
    \item 言語
\end{itemize}

で、\cite{legislationCommittee25minute2000}のように出力されます。
著者名やウェブサイト名は分かれば入れておいた方が良いと思います。

\begin{figure}
    \centering
    \includegraphics[width=0.8\linewidth]{./fig/webpage.png}
    \caption{Webページの入力例。}
    \label{fig:webpage}
\end{figure}

なお、著者名や公開年がわからない場合もあり、
その場合には図\ref{fig:adh}のようになると思います。
この場合の引用は\cite{googleADH}のように出力されます。

\begin{figure}
    \centering
    \includegraphics[width=0.8\linewidth]{./fig/adh.png}
    \caption{著者名や公開年がわからない場合のウェブページの入力例。}
    \label{fig:adh}
\end{figure}


\section{コレクションのエクスポートとOverleafへの取り込み}
\label{sec:export}
コレクションを右クリックして「コレクションをエクスポート...」を選択すると
図\ref{fig:export}のダイアログボックスが表示されます。
フォーマットを「Better BibLaTeX」に、
Keep updatedにチェックを入れてOKをクリックするとエクスポート先を訊かれるので
適宜選択すれば、コレクションがbibファイルとして保存されます。
Keep updatedをオンにしているので、これ以降はコレクションの内容を更新すると
自動でエクスポート先のbibファイルも更新されます。

完成したbibファイルはOverleafの左ペインにあるアップロードアイコン(図\ref{fig:upload})
をクリックすればアップロード可能です。

\begin{figure}
    \centering
    \includegraphics[width=0.65\linewidth]{./fig/export.png}
    \caption{コレクションエクスポート時に表示されるダイアログボックス。}
    \label{fig:export}
\end{figure}

\begin{figure}
    \centering
    \includegraphics[width=0.65\linewidth]{./fig/upload.png}
    \caption{Overleafへのアップロードアイコン。}
    \label{fig:upload}
\end{figure}


\bibliography{ipsj_eip_dummy}
\bibliographystyle{ipsjunsrt_eip}

\end{document}
